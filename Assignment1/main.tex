%iffalse
\let\negmedspace\undefined
\let\negthickspace\undefined
\documentclass[journal,12pt,twocolumn]{IEEEtran}
\usepackage{cite}
\usepackage{amsmath,amssymb,amsfonts,amsthm}
\usepackage{algorithmic}
\usepackage{graphicx}
\usepackage{textcomp}
\usepackage{xcolor}
\usepackage{txfonts}
\usepackage{listings}
\usepackage{enumitem}
\usepackage{mathtools}
\usepackage{gensymb}
\usepackage{comment}
\usepackage[breaklinks=true]{hyperref}
\usepackage{tkz-euclide} 
\usepackage{listings}
\usepackage{gvv}    
\usepackage{multicol}

%\def\inputGnumericTable{}                                 
\usepackage[latin1]{inputenc}                                
\usepackage{color}                                            
\usepackage{array}                                            
\usepackage{longtable}                                       
\usepackage{calc}                                             
\usepackage{multirow}                                         
\usepackage{hhline}                                           
\usepackage{ifthen}                                           
\usepackage{lscape}
\usepackage{tabularx}
\usepackage{array}
\usepackage{float}


\newtheorem{theorem}{Theorem}[section]
\newtheorem{problem}{Problem}
\newtheorem{proposition}{Proposition}[section]
\newtheorem{lemma}{Lemma}[section]
\newtheorem{corollary}[theorem]{Corollary}
\newtheorem{example}{Example}[section]
\newtheorem{definition}[problem]{Definition}
\newcommand{\BEQA}{\begin{eqnarray}}
\newcommand{\EEQA}{\end{eqnarray}}
\newcommand{\define}{\stackrel{\triangle}{=}}
\theoremstyle{remark}
\newtheorem{rem}{Remark}

% Marks the beginning of the document

\title{CHAPTER 3 Quadratic Equation and Inequations(Inequalities)}
\date{}
\begin{document}
\bibliographystyle{IEEEtran}
\vspace{3cm}

\author{ai24btech11022 - Pabbuleti Venkata Charan Teja}
\maketitle
\newpage
\bigskip

\renewcommand{\thefigure}{\theenumi}
\renewcommand{\thetable}{\theenumi}
\maketitle{Section-A JEE Advanced / IIT JEE}

\maketitle{G : Comprehension Based Questions}
$$PASSAGE-1$$
Let $p,q$ be integers and let $\alpha$,$\beta$ be the roots of the equation,
$x^{2}-x+1$, where $\alpha$$\neq$$\beta$. For $n=0,1,2\dots$, let ${a_{n}=p\alpha^{n}+q\beta^{n}}$

FACT : If $a$ and $b$ are rational numbers and $a+b\sqrt{5}=0$, then $a=0=b.s$
\begin{enumerate}
    
\item
$a_{12}=$ \hfill(JEE Adv.2017)


(a) $a_{11}-a_{10}$

(b) $a_{11}+a_{10}$

(c) $2a_{11}+a_{10}$

(d) $a_{11}+2a_{10}$
\item 
If $a_{4}=28$ and $p+2q$=\hfill(JEE Adv.2017)

\end{enumerate}

\begin{multicols}{2}

\begin{enumerate}

\item[(a)] 
21
\item[(c)] 
7
\item[(b)] 
14
\item[(d)] 
12

\end{enumerate}

\end{multicols}

\maketitle{H : Assertion \& Reason Type Questions}
\begin{enumerate}
\item 
Let $a,b,c,p,q$ be real numbers. Suppose $\alpha$,$\beta$ are the roots of equation ${x^{2}+2px+q=0}$ and $\alpha,\frac{1}{\beta}$ are the roots of  the equation ${ax^{2}+2bx+c=0}$, where $\beta^{2}\not\in\{-1,0,1\}$ \hfill(2008)

\end{enumerate}

STATEMENT-1 : $\brak{p^{2}-q}\brak{b^{2}-c}\geq0$

and

STATEMENT-2 : $b\neq pa$ or $c\neq qa$
\begin{enumerate}
\item[(a)]
STATEMENT-1 is true, STATEMENT-2 is true; STATEMENT-2 is a correct explanation for STATEMENT-1
\item[(b)]
STATEMENT-1 is true, STATEMENT-2 is true; STATEMENT-2 is NOT a correct explanation for STATEMENT-1
\item[(c)]
STATEMENT-1 is true,STATEMENT-2 is false
\item[(d)]
STATEMENT-1 is false,STATEMENT-2 is true 

\end{enumerate}

\maketitle{I : INTEGER VALUE CORRECT TYPE}
\begin{enumerate}

\item 
Let $\brak{x,y,z}$ be points with integer coordinates satisfying the system of homogeneous equations : $$3x-y-z=0$$$$-3x+z=0$$$$-3x+2y+z=0$$

Then the number of such points for which ${x^{2}+y^{2}+z^{2}\leq100}$ is \hfill(2009)

\item 
The smallest value of $k$, for which both of the roots of the equation $$x^{2}-8kx+16\brak{k^{2}-k+1}=0$$

are real, distinct and have values at least 4, is 

                                \hfill(2009)

\item 
The minimum value of the sum of real numbers $a^{-5},a^{-4},3a^{-3},1,a^{8}$ and $a^{10}$ where $a>0$ is

\hfill(2011)

\item 
The number of distinct real roots of $$x^{4}-4x^{3}+12x^{2}+x-1=0$$ is\hfill(2011)
\end{enumerate}
\maketitle{Section-B JEE Main / AIEEE}
\begin{enumerate}
\item 
If $\alpha\neq\beta$ but ${\alpha^{2}=5\alpha-3}$ and ${\beta^{2}=5\beta-3}$ then the equation having $\frac{\alpha}{\beta}$ and $\frac{\beta}{\alpha}$ as its roots is\hfill(2002)

\begin{multicols}{2}

\begin{enumerate}

\item[(a)]$3x^{2}-19x+3=0$

\item[(c)]$3x^{2}-19x-3=0$

\item[(b)] $3x^{2}+19x-3=0$

\item[(d)] $x^{2}-5x+3=0$

\end{enumerate}

\end{multicols}

\item 
Difference between the corresponding roots of ${x^{2}+ax+b=0}$ and ${x^{2}+bx+a=0}$ is same and $a\not=b$, then \hfill(2002)
\begin{multicols}{2}
    
\begin{enumerate}

\item[(a)] $a+b+4=0$

\item[(c)] $a-b-4=0$

\item[(b)] $a+b-4=0$

\item[(d)]$a-b+4=0$
\end{enumerate}
\end{multicols}
\item 
Product of real roots of the equation ${t^{2}x^{2}+\abs{x}+9=0}$\hfill(2002)
\begin{multicols}{2}
\begin{enumerate}

\item[(a)] is always positive

\item[(c)] does not exist

\item[(b)] is always negative

\item[(d)] none of these
\end{enumerate}
\end{multicols}
\item 
If $p$ and $q$ are the roots of the equation ${x^{2}+px+q=0}$, then\hfill(2002)
\begin{multicols}{2}
\begin{enumerate}

\item[(a)] $p=1,q=2$

\item[(c)]$p=-2,q=0$

\item[(b)] $p=0,q=1$

\item[(d)] $p=-2,q=1$
\end{enumerate}
\end{multicols}
\item 
If $a,b,c$ are distinct $+ve$ real numbers and ${a^{2}+b^{2}+c^{2}=1}$ then ${ab+bc+ca}$ is\hfill(2002)
\begin{multicols}{2}
\begin{enumerate}

\item[(a)] less than 1

\item[(c)] greater than 1

\item[(b)] equal to 1

\item[(d)] any real no.
\end{enumerate}
\end{multicols}
\item 
If the sum of the roots of the quadratic equation ${ax^{2}+bx+c=0}$ is equal to the sum of the squares of their reciprocals, then $\frac{a}{c},\frac{b}{a},\frac{c}{b}$ are in 

\hfill(2003)

(a) Arithmetic - Geometric Progression

(b) Arithmetic Progression

(c) Geometric Progression

(d) Harmonic Progression
\item 
The value of '$a$' for which one root of the quadratic equation ${\brak{a^{2}-5a+3}x^{2}+\brak{3a-1}x+2=0}$ is twice as large as the other is \hfill(2003)
\begin{multicols}{2}

\begin{enumerate}
\item[(a)] $-\frac{1}{3}$

\item[(c)] $-\frac{2}{3}$

\item[(b)] $\frac{2}{3}$

\item[(d)] $\frac{1}{3}$
\end{enumerate}

\end{multicols}

\end{enumerate}
\end{document}