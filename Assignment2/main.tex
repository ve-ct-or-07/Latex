 %iffalse
\let\negmedspace\undefined
\let\negthickspace\undefined
\documentclass[journal,12pt,twocolumn]{IEEEtran}
\usepackage{cite}
\usepackage{amsmath,amssymb,amsfonts,amsthm}
\usepackage{algorithmic}
\usepackage{graphicx}
\usepackage{textcomp}
\usepackage{xcolor}
\usepackage{txfonts}
\usepackage{listings}
\usepackage{enumitem}
\usepackage{mathtools}
\usepackage{gensymb}
\usepackage{comment}
\usepackage[breaklinks=true]{hyperref}
\usepackage{tkz-euclide} 
\usepackage{listings}
\usepackage{gvv}                       

%\def\inputGnumericTable{}                                 
\usepackage[latin1]{inputenc}                                
\usepackage{color}                                            
\usepackage{array}                                            
\usepackage{longtable}                                       
\usepackage{calc}                                             
\usepackage{multirow}                                         
\usepackage{hhline}                                           
\usepackage{ifthen}                                           
\usepackage{lscape}
\usepackage{tabularx}
\usepackage{array}
\usepackage{float}
\usepackage{multicol}


\newtheorem{theorem}{Theorem}[section]
\newtheorem{problem}{Problem}
\newtheorem{proposition}{Proposition}[section]
\newtheorem{lemma}{Lemma}[section]
\newtheorem{corollary}[theorem]{Corollary}
\newtheorem{example}{Example}[section]
\newtheorem{definition}[problem]{Definition}
\newcommand{\BEQA}{\begin{eqnarray}}
\newcommand{\EEQA}{\end{eqnarray}}
\newcommand{\define}{\stackrel{\triangle}{=}}
\theoremstyle{remark}
\newtheorem{rem}{Remark}
\parindent 0px
% Marks the beginning of the document
\begin{document}
\bibliographystyle{IEEEtran}
\vspace{3cm}

\title{CHAPTER 12 Differentiation}
\author{ai24btech11022 - Pabbuleti Venkata Charan Teja}
\maketitle{Section-A JEE Advanced / IIT-JEE}

\maketitle{A : Fill in the Blanks}
\begin{enumerate}
\item[7)]
If $f\brak{x}$ is a twice differentiable function and given that $f\brak{1}=1,f\brak{2}=4,f\brak{3}=9$, then 

\hfill{(2005S)}

(a) $f^{\prime\prime}\brak{x}=2$ $,\forall$ $x\in\brak{1,3}$

(b) $f^{\prime\prime}\brak{x}=f^{\prime}\brak{x}=5$ for some $x\in\brak{2,3}$

(c) $f^{\prime\prime}\brak{x}=3$ $,\forall$ $x\in\brak{2,3}$

(d) $f^{\prime\prime}\brak{x}=2$ for some $x\in\brak{1,3}$

\item[8)]
$\frac{d^{2}x}{dy^{2}}$ \hfill{(2007-3 marks)}

\begin{multicols}{2}
\begin{enumerate}
\item[(a)]$\brak{\frac{d^{2}y}{dx^{2}}}^{-1}$
\item[(c)]$\brak{\frac{d^{2}y}{dx^{2}}}\brak{\frac{dy}{dx}}^{-2}$
\item[(b)]$-\brak{\frac{d^{2}y}{dx^{2}}}^{-1}\brak{\frac{dy}{dx}}^{-3}$
\item[(d)]$-\brak{\frac{d^{2}y}{dx^{2}}}\brak{\frac{dy}{dx}}^{-3}$
\end{enumerate}    
\end{multicols}

\item[9)]
Let $g\brak{x}=\text{log}f\brak{x}$ is twice differentiable positive function on $\brak{0,\infty}$ such that ${f\brak{x+1}=xf\brak{x}}$. Then, for $N=1,2,3,\dots$ 

\hfill{(2008)}

$g^{\prime\prime}\brak{N+\frac{1}{2}}-g^{\prime\prime}\brak{\frac{1}{2}}=$

(a) $-4 \cbrak{1+\frac{1}{9}+\frac{1}{25}+\dots+\frac{1}{\brak{2N-1}^{2}}}$

(b) $4\cbrak{1+\frac{1}{9}+\frac{1}{25}+\dots+\frac{1}{\brak{2N-1}^{2}}}$

(c) $-4\cbrak{1+\frac{1}{9}+\frac{1}{25}+\dots+\frac{1}{\brak{2N+1}^{2}}}$

(d) $4\cbrak{1+\frac{1}{9}+\frac{1}{25}+\dots+\frac{1}{\brak{2N+1}^{2}}}$

\item[10)]
Let $f\,:\,\sbrak{0,2}\to\mathbb{R}$ be a function which is continuous on $\sbrak{0,2}$ and is differentiable on $\brak{0,2}$ with $f\brak{0}=1$. Let $F\brak{x}=\int \limits_{0}^{x^{2}}f\brak{\sqrt{t}}dt$ for $x\in\sbrak{0,2}$. If $F^{\prime}\brak{x}=f^{\prime}\brak{x}$ for all ${x\in\brak{0,2}}$ then $F\brak{2}$ equals \hfill{(JEE Adv. 2014)}

\begin{multicols}{2}
\begin{enumerate}
\item[(a)]$e^{2}-1$
\item[(c)]$e-1$
\item[(b)]$e^{4}-1$
\item[(d)]$e^{4}$
\end{enumerate} 
\end{multicols}
\end{enumerate}

\maketitle{D : MCQs with One or More than One Correct}

\begin{enumerate}

\item
Let $f\,:\,\mathbb{R}\to\mathbb{R}$, $g\,:\,\mathbb{R}\to\mathbb{R}$ and $h\,:\,\mathbb{R}\to\mathbb{R}$ be differentiable functions such that ${f\brak{x}=x^{3}+3x+2}$, $g\brak{f\brak{x}}=x$ and $h\brak{g\brak{g\brak{x}}}=x$, $\forall x\in\mathbb{R}$. Then 

\hfill{(JEE Adv. 2016)}

\begin{multicols}{2}
\begin{enumerate}

\item[(a)]$g^{\prime}\brak{2}=\frac{1}{15}$
\item[(c)]$h\brak{0}=16$
\item[(b)]$h^{\prime}\brak{1}=666$
\item[(d)]$h\brak{g\brak{3}}=36$ 

\end{enumerate}
\end{multicols}

\item 
For every twice differential function ${f\,:\,\mathbb{R}\to\sbrak{-2,2}}$ with $\brak{f\brak{0}}^{2}+\brak{f^{\prime}\brak{0}}^{2}=85$, which of the following statement(s) is(are) TRUE?

\hfill{(JEE Adv. 2018)}

(a) There exist $r,s\in\mathbb{R}$, where $r<s$, such that $f$ is one-one on the open interval $\brak{r,s}$

(b) There  exists $x_{0}\in\brak{-4,0}$ such that $\abs{f^{\prime}\brak{x_{0}}}\leq1$

(c) $\lim \limits_{x\to\infty}f\brak{x}=1$

(d) There exists $\alpha\in\brak{-4,4}$ such that ${f\brak{\alpha}+f^{\prime\prime}\brak{\alpha}=0}$ and ${f^{\prime}\brak{\alpha}\not=0}$

\item 
For any positive integer $n$, define ${f_{n}\brak{x}=\sum\limits_{j=1}^{n}\tan^{-1}\brak{\frac{1}{1+\brak{x+j}\brak{x+j-1}}},\forall\,x\in\brak{0,\infty}}$

Here, the inverse trigonometric function $\tan^{-1}x$ assumes values in $\brak{-\frac{\pi}{2},\frac{\pi}{2}}$.

Then, which of the following statement(s) is (are) TRUE? \hfill{(JEE Adv. 2018)}

(a) $\sum\limits_{j=1}^{5}\tan^{2}\brak{f_{j}\brak{0}}=55$

(b) $\sum\limits_{j=1}^{10}\brak{1+f_{j}^{\prime}\brak{0}}\brak{\sec^{2}\brak{f_{j}\brak{0}}}=10$

(c) For any fixed positive integer $n$, $\lim\limits_{x\to\infty}\tan\brak{f_{n}\brak{x}}=\frac{1}{n}$

(d) For any fixed positive integer $n$, $\lim\limits_{x\to\infty}\sec^{2}\brak{f_{n}\brak{x}}=1$

\item 
Let $f\,:\,\brak{0,\pi}\to\mathbb{R}$ be a twice differentiable function such that ${\lim\limits_{t\to x}\frac{f\brak{x}\sin t-f\brak{t}\sin x}{t-x}=\sin^{2}x,\,\forall x\in\brak{0,\pi}}$. If $f\brak{\frac{\pi}{6}}=-\frac{\pi}{12}$, then which of the following statement(s) is(are) TRUE? 

\hfill{(JEE Adv. 2018)}

(a) $f\brak{\frac{\pi}{4}}=\frac{\pi}{4\sqrt{2}}$

(b) $f\brak{x}<\frac{x^{4}}{6}-x^{2},\,\forall x\in\brak{0,\pi}$

(c) There exists $\alpha\in\brak{0,\pi}$ such that $f^{\prime}\brak{\alpha}=0$

(d) $f^{\prime\prime}\brak{\frac{\pi}{2}}+f\brak{\frac{\pi}{2}}=0$
\end{enumerate}

\maketitle{E : Subjective Problems}
\begin{enumerate}

\item 
Find the derivative of $\sin\brak{x^{2}+1}$ with respect to $x$ from first principle. \hfill{(1978)}

\item 
Find the derivative of

\begin{equation}
f\brak{x} =
\begin{cases}
\frac{x-1}{2x^{2}-7x+5} & when\,x\not=1\\

-\frac{1}{3} & when\,x=1\\
\end{cases}
\end{equation}

at $x=1$ \hfill{(1979)}

\item 
Given $y=\frac{5x}{3\sqrt{\brak{1-x^{2}}}}+\cos^{2}\brak{2x+1}$. Find $\frac{dy}{dx}$

\hfill{(1980)}

\item 
Let $y=e^{x\sin x^{3}}+\brak{\tan x}^{x}$. Find $\frac{dy}{dx}$ 

\hfill{(1981 - 2 Marks)}

\end{enumerate}


\end{document}